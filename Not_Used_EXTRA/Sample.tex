\section{Analytical Framework}
This chapter contains the theoretical discussion of the several frameworks for assessing sustainable innovations. Firstly in \autoref{sec:Comp} three main frameworks namely: Functions of Innovation systems (FIS), adapted FIS as proposed by Van Alphen and the Strategic Niche Management (SNM) are introduced and compared. In the end of the section, one of the frameworks which is most suitable for this study is chosen. A detailed description of the chosen framework can be found in \autoref{sec:FIS} and in \autoref{sec:MLP} a multi level perspective for sustainable innovations is introduced. Finally, in \autoref{sec:Indicators} the indicators for the chosen framework are enlisted along with the reason for choosing them.

%%%Description and comparison of FIS and SNM frameworks: please be more explicit here and mention for SNM and FIS Van Alphen the specific elements/functions.
\subsection{ SNM, FIS and adapted FIS : A comparison}
\label{sec:Comp}
Several approaches to analyze a sustainable innovation exist. In this section the SNM, FIS and the adapted FIS will be introduced and in the end compared to each other.

\textbf{Strategic Niche Management (SNM)}\\
New innovations are generally firstly applied in small niches where they are allowed to experiment and adapt in order to improve and grow. They grow out of one niche and into another niche, where in the end large scale diffusion should take off. SNM is there to analyze this development and transition before the diffusion \citep{Laak2007}. It is an evolutionary economics tool that can identify successes and failures in the development of sustainable technologies \citep{Eijck2008}. It concerns the use and user behaviour of the development. The focus with SNM is with identifying problems that stimulated development of the technology, together with, to what extent the actors influenced the development within a niche \citep{Raven2005}. It can be applied on local, regional and national level.\\
\noindent The analysis SNM performs makes use of three processes that happen in the a niche. The processes are, according to \cite{Laak2007} the following: 
\begin{enumerate}
    \item The voicing and shaping of expectations
    \item The building of a social network
    \item A good learning process
\end{enumerate}
The interaction between these processes give insights as to why a technological innovation fails or succeeds. 
It is argued by \cite{Raven2005} that a fourth factor can be identified that influences successes and failure. This is the dynamics of social technical regimes. However, it is seen that this factor is only active indirectly \citep{Laak2007}.

\textbf{Functions of Innovation System (FIS)}\\
The FIS approach also aims to identify the failures and successes in the development of a technological innovation. However, FIS is more focused on innovations that are still in the research stage. It also focuses more on the development of the knowledge over time instead of focusing on the use and users perspective like SNM. In other words: the difficulties encountered when a new technology is brought to the market are analyzed \citep{Negro2007} instead of how the evolution such a technology undergoes within protected environments. The framework makes use of the idea that the innovations are consisting of several phases, namely the formative phase and the diffusion phase \citep{Negro2007}. In order to analyze these phases, FIS makes use of seven functions of innovation. When these functions are fulfilled correctly, the new technology should not experience difficulties to reach large scale diffusion \citep{Kamp2008}. The functions are further explained in \ref{sec:FIS}. FIS can be applied on region, country or international level. 

\textbf{Adapted FIS}\\
The FIS approach described above, is seen to be applied in mainly developed countries. According to \cite{alphen_hekkert_sark_2008} the `conventional FIS approach does not give adequate results for developing countries since more and different external factors play a role in the diffusion of new technologies. This means that the functions of innovations, that are defined for industrialized economies, need to be altered in such a way that they can be applied to economically developing countries  \citep{alphen_hekkert_sark_2008}. In that way, technology transitions can be analyzed and bottlenecks can be identified. \cite{alphen_hekkert_sark_2008} argued that the functions for non industrialized economies should be the following seven:
\begin{enumerate}
    \item Creating adaptive capacity
    \item Knowledge diffusion through networks
    \item Demand articulation
    \item Creation of legitimacy/counteract resistance to change
    \item Resource mobilization
    \item Market formation
    \item Entrepreneurial activities
\end{enumerate}
From these functions it is seen that different functions need to be fulfilled for developing countries to be able to succeed in large scale diffusion of a new technology. \\

Now that the different analytical frameworks are introduced, they can be compared. It is seen that the main differences between SNM and FIS are that SNM is focussed on use and user interactions, how they create development and diffusion, whereas FIS is mainly focused on the knowledge development of the technology. Another important difference is that SNM is applied to technologies that are already introduced to a (niche) market, where FIS is analyzing the technologies before market introduction. The last main difference is the level to which it is applied, SNM is applied on local, regional and national level, and FIS is applied on regional, country and international level. 

Comparing FIS and the adapted FIS results in a difference in development state of the country itself. Since the functions of innovation are not truly suitable for developing countries, FIS is less suitable and the adapted FIS approach is used. When SNM and adapted FIS are compared, the differences mentioned above are applicable and also that adapted FIS is applied for developing countries where SNM is not. Now that the different approach options are identified and discussed, one can be chosen for the analysis of geothermal Energy in the horticulture sector in the Netherlands. 

\subsection{Approach for this study: FIS  }
\label{sec:FIS}
Based on the discussion points as explained in \autoref{sec:Comp}, the FIS approach is used to perform the analysis for this study. In this section, FIS is further explained along with the several functions that are critical to analyse the growth and diffusion of an innovation system. In this study, seven key functions as discussed by \cite{FIS2009} are used to understand how the innovation is currently performing and also to predict its growth in the upcoming years. In addition to this, the functional pattern of the technology's development is observed through these system functions and areas that need attention is made visible \citep{Kamp2008}. The functions that are used in this study are \citep{FIS2009}:

\textbf{Entrepreneurial Activities:} This function deals with the entrepreneurial activities that are taken up in relation to the innovation. This is very important as the new idea is translated into a working business model which requires extra efforts as opposed to a standardized technology.

\textbf{Knowledge Development (learning):} This function deals with the expansion of the existing knowledge about the innovation and the principles behind it. A continuous research and development is rudimentary for any innovation process. This is usually done in a controlled environment at the incubation stage but later translated into learning by practicing it on field and the response to dynamic situations are then noted which again adds to the knowledge base.

\textbf{Knowledge Diffusion through Networks:} Developing knowledge alone is not sufficient but it has to be transferred to other parties such as institution stakeholders, like-minded investors in the field and also the public. This makes sure that there is a cross exchange of valuable information which helps the government to make an enabling environment for the technology through its bylaws and action plans. On the other hand, when other parties also have a common goal (the innovation), it is important to work together which makes the knowledge development process to go one step further. 

\textbf{Guidance of the Search:} This function refers to setting the scene for the technology and what it aims to achieve but in a clearer perspective and helps to propel in a particular direction. This, in itself is not a driver, but it rather enables the other system functions, for example market formation or resource mobilization. In simple terms, it makes the general audience and involved parties more informed that there is a wider interest and helps them to converge.

\textbf{Market Formation:} It is a well known fact that products/solutions fail when there is no demand for it. This is where creation of market is essential for a sustainable innovation to be actively assimilated. The fact that there are established alternatives with higher credibility and customer base is a concern for newly emerging solutions. Thus, there needs to be an incubation setting for the innovation either by institutional stakeholders or other parties. Subsides, tax exemptions or regulations are ideal examples for this function.

\textbf{Resource Mobilisation:} The resources that are essential for the innovative solution are crucial for it to be expand the coverage. This means, the men, money and material resources for the innovation need to be planned accordingly. Whether the resources are local or external and the skills required for implementation and upkeep are some of the crucial points under this function.

\textbf{Creation of legitimacy/Support from advocacy coalitions:} For the technological innovation to be truly successful, it has to integrate with the policy regulations and frameworks as defined by the institutional stakeholders. There is always political opposition in these situations as it can be observed in certain countries where climate change is not a matter of urgency for the ruling government and causes extensive debates. This can potentially hamper the uptake of sustainable energy solutions in these country. Thus there needs to be clear legislation and policy frameworks with respect to the innovation and its underlying values.


It is important to note that each of these functions in itself can not ensure the success of the technology but is the positive interaction between these functions which then will reinforce the dynamics of the system diffusion \citep{FIS2009}. It is also possible that some of the functions might develop negatively and can lead to deterioration of the whole diffusion process. This is known as the motors of change and an example of such a feedback loop can be seen in \autoref{fig:FL} \citep{hekkert_suurs_negro_kuhlmann_smits_2007}. 
\begin{figure}[H]
    \centering
    \includegraphics[scale = 0.5]{Images/Step3/FL.png}
    \caption{An example of motors of change, source : \citep{hekkert_suurs_negro_kuhlmann_smits_2007}}
    \label{fig:FL}
\end{figure}
In addition to that, based on the insights from the FIS analysis, the areas that need attention can be realised by the developers/promoters of the innovation as explained by \cite{alphen_hekkert_sark_2008}. A normative approach can also be adopted by the institutional stakeholders in order to set the transition context in terms of governance as explained by \cite{RePEc:eee:respol:v:34:y:2005:i:10:p:1491-1510}. 

\subsubsection{ Indicators}
\label{sec:Indicators}
Based on the functions that are described in the previous section, it is important to define clear indicators which are visible/measurable. This section provides a list of indicators which will be used for the analysis later on in the study.  Several literature is used for this purpose and the chosen indicators are summarized in \autoref{tab:indica} \citep{hekkert_suurs_negro_kuhlmann_smits_2007} \citep{Kamp2008} \citep{FIS2009}.

\begin{table}[H]
\caption{List of indicators for the functions that will be used for the assessment}
\label{tab:indica}
\resizebox{\textwidth}{!}{%
\begin{tabular}{l|l|l|l}
  & Function                             & Chosen indicators                                                                                                                                             & Source                   \\ \hline
1 & Entrepreneurial Activities           & \begin{tabular}[c]{@{}l@{}}- New entrants into the market\\ - Incumbent companies that diversify to geothermal\end{tabular}                                   & \citep{Kamp2008}             \\ \hline
2 & Knowledge Development                & \begin{tabular}[c]{@{}l@{}}- R\&D projects\\ - Patents\\ - Investment in R\&D\\ - Learning curves with respect to geothermal technology\end{tabular}              & \citep{hekkert_suurs_negro_kuhlmann_smits_2007}   \\ \hline
3 & Knowledge Diffusion through Networks & \begin{tabular}[c]{@{}l@{}}- Conferences devoted to geothermal energy\\ - Number of workshops conducted\\ - Network size and intensity over time\end{tabular} & \citep{hekkert_suurs_negro_kuhlmann_smits_2007}   \\ \hline
4 & Guidance of the Search               & \begin{tabular}[c]{@{}l@{}}- National goals\\ - Policy and regulations\end{tabular}                                                                           & \citep{Kamp2008}             \\ \hline
5 & Market Formation                     & \begin{tabular}[c]{@{}l@{}}- Number of niche markets introduced\\ - Specific tax regimes and environmental standards\\ for new technologies \end{tabular}    & \citep{hekkert_suurs_negro_kuhlmann_smits_2007}   \\ \hline
6 & Resource Mobilisation                & \begin{tabular}[c]{@{}l@{}}- Subsidies\\ - Investments\end{tabular}                                                                                           & \citep{FIS2009} \\ \hline
7 & Support from advocacy coalitions               & \begin{tabular}[c]{@{}l@{}}- Lobby by agents to improve technical, \\ institutional and financial conditions\end{tabular}                                     & \citep{FIS2009} 
\end{tabular}%
}
\end{table}


\subsubsection{ Multi Level Perspective (MLP)}
\label{sec:MLP}
MLP is a tool to create understanding about the dynamics when a change in a socio-technical system is happening. It does not just describe the dynamics found in a specific case, but it subtracts analytical tools and strategies from several cases in order to create understanding of such changes \citep{Geels2002}. \\
\noindent MLP identifies three levels \citep{Raven2005}: 
\begin{enumerate}
    \item The level of niches
    \item The level of socio-technical regime
    \item The level of socio-technical landscape
\end{enumerate}
The level of niches is an important one for emerging technologies. Within a niche, a technology is allowed to experiment and while doing so, it is protected. The technology is allowed to adapt to its environment within a niche in order to develop and evolve. Niches are very important for technological innovations. A niche functions as an incumbation area where drastic adaptions happen and where they get the opportunity to form important social networks which they will need later on \citep{Geels2002}.

The level of socio-technical regime indicates the different actors involved in a specific technology system that uses the same way of interpreting and are following the same rules, and use a same way of idea collection. It consists of the technical actors like engineers as well as policy makers and consumers as the social actors. Thus, the actors from different perspectives are all aligned due to using the same set of cognitive rules, and because of this it provides structure for developing technologies \citep{Geels2002} \citep{Raven2005}.

The third level, the level of socio-technical (ST) landscape is an extension of the first level. The ST landscape is the same ST regime of level 1, but now includes external factors. This means that actors that dealing with actors that are not aligned is needed. The ST landscape is a more complex actor network and contains a set of heterogeneous factors, like economic growth, political instability, et cetera \citep{Geels2002}.

\begin{figure}[H]
    \centering
    \includegraphics[scale = 0.8]{Images/MLP.png}
    \caption{Multi level perspective visualized, received from \citep{Geels2002}.}
    \label{fig:MLP}
\end{figure}
In figure \ref{fig:MLP} the MLP is visualized. In the picture, the three different levels are clearly indicated. Also, it is seen from this picture that the different levels are connected to each other. It becomes more clear that the technology starts in a niche. Than diffuses to the ST regime where phenomenon of trajectories exists and stability is assured since technological development is a common. In the end it diffuses to the ST landscape, here it encounters difficulties to overcome due to the external actors that adapt more slowly \citep{Geels2002}. \\
To summarize, MLP is to show that success of an emerging technology is not only governed by that what happens in a regime, but also by innovations within the ST regime and at the landscape level \citep{Geels2002}.
